
\chapter{Background}
\label{ch:Background}

% In this chapter, we introduce and define common data structures that are used to represent graphs in memory.
% Our choice of in-memory graph data structure naturally leads into questions about how the graph's elements
% (vertices, edges) should be laid out, or ordered, within the constraints of the data structure we use to store them, which is the main focus of this thesis.

% We begin with Table \ref{tab:notation} which defines a common ground of 
% % conventional 
% symbols and constants that we 
% will 
% % continuously 
% refer to in this thesis. 
% We touch on the conventional data structures that have been used to represent graphs: \textbf{Adjacency List} and \textbf{Adjacency Matrix}, and how their inadequacies in representing large-scale, real-world, sparse graphs, has prompted the development of compressed graph representations: \ac{CSR} / \ac{CSC}. The widespread utilization of the \ac{CSR} as the de-facto data structure for in-memory \ac{GPS} has raised some important challenges, which \textit{graph ordering} has attempted to ameliorate and resolve.
% We define graph ordering, and specialize our definition by making the distinction between two common graph ordering methods: Vertex and Edge ordering
% %  and describe and categorize a selection of vertex and edge ordering techniques.
% We conclude this chapter with a discussion of different \textit{modes} of graph computation, and how the hurdles we describe in our introduction of the \ac{CSR} representation are further compounded in the parallel setting.


We introduce common data structures that are used to represent graphs in memory. Once a data structure is selected, how we \textit{layout}, or \textit{order} the graph's elements and data within the chosen data structure is the main focus of this thesis. We define our notation in Table \ref{tab:notation}. We review the conventional graph data structures: Adjacency List and Adjacency Matrix, and show how they are inadequate for representing large-scale, real-world, sparse graphs. To address these limitations, compressed graph representations like \ac{CSR} and \ac{CSC} are used. 
% The \ac{CSR} has become the de-facto data structure for in-memory \ac{GPS}, but using it for graph analytics poses challenges that are further exacerbated in the parallel setting.
We introduce the different modes of parallel graph computation and illustrate the challenges that arise when using the \ac{CSR} representation for parallel graph processing. Finally, we show how graph ordering can address these challenges. We conclude by defining graph ordering and distinguish between two common methods: Vertex and Edge ordering. 
\clearpage
% \section{Notation}
\setlength{\arrayrulewidth}{0.5mm}
\setlength{\tabcolsep}{18pt}
\renewcommand{\arraystretch}{1.5}

% \begin{lstlisting}
% Expand left and right margins by 0.75in
% Must adjust the perceived column width for LaTeX to get with it.
% \addtolength{\columnwidth}{1.5in}
% \[ an extra long math formula \]

\begin{table}
    \centering
    \begin{adjustwidth}{-0.75in}{-0.75in}

        \begin{tabular}{ |p{2cm}|p{6cm}|p{5cm}|  }
            % \hline
            % \multicolumn{3}{|c|}{Country List}                         \\
            \hline
            \textbf{Symbol}  & \textbf{Definition}                                                                                                 & \textbf{Notes}                \\
            \hline
            $V$              & Vertex Set                                                                                                          & -                             \\
            \hline
            $E$              & Edge Set                                                                                                            & -                             \\
            \hline

            $G(V, E)$        & A graph $G$, with vertex set $V$, and edge set $E$                                                                  & -                             \\
            \hline

            $G(V, E, w(E))$  & A weighted graph $G$, with vertex set $V$, edge set $E$, and a weight function $w: (V \times V) \mapsto \mathbb{R}$ & -                             \\
            \hline


            $B(V_1, V_2, E)$ & A bipartite graph $B$, with vertex sets $V_1, V_2$, and edge set $E$                                                & -                             \\
            \hline

            $n$              & The number of vertices in a graph                                                                                   & $|V|$                   \\
            \hline

            $m$              & The number of edges in a graph                                                                                      & $|E|$                   \\
            \hline

            $(u, v)$         & The edge between vertices $u, v$                                                                                    & May be directed or undirected \\
            \hline

            $A$              & The adjacency matrix of the directed graph, $G$                                                                     &                               \\
            \hline
            $n^{+/-}(u)$       & The out/in-neighbourhood of vertex $u$                                                                                 &                               \\
            \hline
            $n(u)$       & The neighbourhood of vertex $u$                                                                                  &                               \\
            \hline
            $deg^{+/-}(u)$     & The out/in-degree of vertex $u$ & $|n^{+}(u)|$/$|n^{-}(u)| $           \\
            \hline
            $deg(u)$     & The degree of vertex $u$   & $|n(u)|$            \\
            \hline
            $f(G)$     & The fill of a graph & $\frac{m}{n(n-1)}$            \\
            \hline
            $\overline{d}$ & The average degree of the graph & $\frac{m}{n}$            \\
            \hline
        \end{tabular}
        \caption{\label{tab:notation}Notation used in this thesis.}
    \end{adjustwidth}
\end{table}


% \end{lstlisting}


\section{In Memory Graph Data Structures}
% \par{
%     % We introduce ways to represent graphs in memory. We describe the conventional Adjacency matrix and 
%     % list representations of a graph, and how the vertices and edges of the graph are stored using 
%     % these data structures. Next, we introduce the commonly used, efficient   
%     % \ac{CSR}/\ac{CSC}
%     % graph representations, and illustrate challenges that arise from representing large graphs using these
%     % compressed representations. 
%     We introduce various methods for storing graphs in memory, including the traditional Adjacency Matrix and List representations. We explain how vertices and edges are stored using these data structures. We then present the efficient Compressed Sparse Row (CSR) and Compressed Sparse Column (CSC) representations commonly used for large graphs and highlight the challenges that arise when using these compressed representations.
% }


A graph $G(V, E)$ is an ordered pair of a vertex set, $V$, and an edge set $E$. Graphs may be directed and/or weighted. If a graph is directed, $E$ consists of \textit{ordered} pairs of vertices, whereas is $G$ is undirected, edges are \textit{unordered}.
$n$ is the number of vertices in the graph ($|V|$) and $m$ is the number of edges in the graph ($|E|$).
Given a directed graph and an edge $(u,v)$, we say that that $v$ is an \textit{out-neighbour} of $u$, $u$ is an in-neighbour of $v$, and that $u$ and $v$ are adjacent. In this thesis, we will assume graphs are directed.
If $G$ is a weighted graph, a weight function, $w: (V \times V) \mapsto \mathbb{R}$, maps each edge in the graph to a real value. A graph $B$ is said to be bipartite when $V$ is decomposed into two disjoint vertex sets $V_1, V_2$, such that no two vertices within a vertex set are adjacent.
Figure \ref{fig:graph_example}a shows a directed graph, which we will use as a running example to show how the same graph is represented using either an Adjacency Matrix, Adjacency List, or the \ac{CSR} / \ac{CSC} representations.



% \caption{A Directed Graph $G(V, E)$ with \\$V=\{0,1,2,3,4,5,6,7\}$ and \\$E=\{
%         (0, 2),
%         (1, 5),
%         (1, 6),
%         (2, 0),
%         (2, 1),
%         (2, 4),
%         (2, 5),
%         (3, 2),
%         (3, 5),            
%         (4, 5),
%         (5, 2),
%         (5, 4),
%         (5, 6),
%         (6, 5)\}$ and its corresponding boolean, directed adjacency matrix, $A$}
\begin{figure}[!htb]
    \begin{adjustwidth}{-0.75in}{-0.75in}
        \centering
        \includesvg[width=6in]{./ipe_plots/in_mem_graph_structs.svg}
        \caption{
            In-memory Graph Representations
        }
        \label{fig:graph_example}   % label should change
    \end{adjustwidth}
\end{figure}

A graph can be represented using an \textbf{Adjacency Matrix}, $A$ (Figure \ref{fig:graph_example}b), with $A_{i,j}$ corresponding to the edge $(i, j)$. If the graph is unweighted, $A$ is an $n\times n$ boolean matrix, with $A_{i,j} = 1$ indicating the existence of an edge $(i, j)$.
If the graph is weighted, the nonzero values of $A\in\mathbb{R}^{n\times n}$ correspond to the weight function, $w(E)$.

Many real world graphs derived from bioinformatics, social, or hyperlink networks are
\textit{scale-free} \cite{sapco} and \textit{sparse} \cite{danisch2018listing}.
A graph is \textit{scale-free} if the degree distribution of its vertices follows a \textit{power-law}, where a small fraction of vertices make up a disproportionately large fraction of the edges of the graph.
A graph is \textit{sparse} if the number of edges in the graph is substantially smaller than the total number of possible edges in the graph (i.e., $m \ll n(n-1)$). 
Correspondingly, the adjacency matrices of real world graphs are themselves, sparse matrices, with a majority of the entries in the matrix containing zero, or null, values. This observation has led researchers to frame graph processing in the form of matrix operations like \ac{SPMV} \cite{graphmat}. For example, the \textit{out-(in-)degrees} of the vertices in the graph can be computed using the matrix-vector product: $A\mathbf{1}^T$ ($A^T\mathbf{1}^T$). However, representing real world graphs as sparse matrices is usually infeasible, due to the $O(n^2)$ memory footprint of the entire adjacency matrix.

Alternatively, a graph can be represented using an \textbf{Adjacency List} (Figure \ref{fig:graph_example}c), with the vertices of the graph stored in an array, where the vertices are sorted by their ascending Vertex ID. The neighbourhood of vertex $u$ (either $n^{+}(u)$ or $n^{-}(u)$ for the out- or in-neighbourhood of $u$) is usually stored in a linked list, and $u$'s neighbours are sorted by ascending Vertex ID. The memory requirements of the adjacency list ($O(n + m)$) are lesser than that of the adjacency matrix, which is particularly crucial when dealing with sparse graphs. However, the use of pointers to link vertex neighbourhoods raises an issue.
Graph algorithms typically rely on the repeated traversal of the neighbourhoods of vertices (e.g., \ac{BFS}, \ac{SSSP}). If the graph is stored using an adjacency list, this traversal will follow neighbouring pointers to iterate over the neighbourhood of each vertex. This raises difficulties for hardware prefetchers, since, assuming the neighbourhood linked list has not been seen in the past, a prefetcher cannot prefetch the data of the next vertex until we've loaded the neighbour pointer of the current vertex. This ``pointer-chasing'' issue hindering performance gains by hardware prefetching can be alleviated by laying out the neighbourhoods in a sorted, contiguous array.
%  which led to the de-facto in-memory graph data structure, the Compressed Sparse Row/Column (\ac{CSR}/\ac{CSC}).

The \textbf{Compressed Sparse Row/Column (\ac{CSR}/\ac{CSC})} representations (Figure \ref{fig:graph_example}d) achieve this data layout by using two arrays to store the vertices and edges of the graph. The \textbf{Offsets Array}, \texttt{OA}, is an array of size $n+1$ that corresponds to the vertices of the graph. Vertices are stored in \texttt{OA} in ascending order of their vertex ID. The \textbf{Neighbours Array}, \texttt{NA}, is an array of size $m$ that corresponds to the edges of the graph. To iterate over the neighbourhood of vertex $u$, we first retrieve the beginning and end offset of $u$'s neighbourhood from \texttt{OA[u]} and \texttt{OA[u + 1]}. These offsets tell us the range of indices in \texttt{NA} that hold the Vertex IDs of the neighbours of $u$. Vertices within a neighbourhood are sorted by ascending order of their vertex ID. If the graph is weighted, an additional \textbf{Weights Array} of size $m$ (\texttt{WA}) stores edge weights. A \ac{CSR} stores the out-edges of the graph, while a \ac{CSC} stores the in-edges of the graph. Using a compressed representation to store the graph achieves the benefits gained by both the adjacency matrix and adjacency list.
First, fast Matrix-Vector operations are enabled, since we can access the nonzero values of row $i$ (stored in \texttt{NA[}\texttt{OA[i]}: \texttt{OA[i + 1]}\texttt{]}) sequentially and in constant time.
Similarly, we no longer require any pointer chasing to iterate over the neighbourhood of vertex $i$ (also stored in \texttt{NA[}\texttt{OA[i]}: \texttt{OA[i + 1]}\texttt{]}).

\section{Memory Access Patterns and Bottlenecks in Graph Processing}
Due to its memory efficiency, the \ac{CSR} representation has become the de-facto data structure for storing graphs in-memory. However, there due to the sparsity of the graph, traditional graph traversal algorithms suffer from \textit{poor locality} when using the \ac{CSR} representation. In fact, prior work has shown that graph algorithms spend a majority of execution time stalled on memory accesses \cite{zhang2016optimizing}. 

Consider Algorithm \ref{alg:algorithm1}, which is an example of 
a common graph traversal, where we iterate over the out-neighbourhoods of the vertices of the graph. For each out-neighbour, we access an array of size $n$, \texttt{v\_data}, that stores application-specific vertex data.
Figure \ref{fig:irregular_data_access} illustrates the irregularity of access of this common graph traversal pattern.
Using this figure as reference, we can distinguish between two different types of locality:
\begin{enumerate}
    \item \textit{Spatial Locality}: a vertex $u$'s neighbours exhibit \textit{spatial} locality if the ID range of its neighbours is consecutive and/or contiguous. Equivalently, we require a minimal number of cache lines to retrieve the out-neighbourhood of $u$. For example, in Figure \ref{fig:irregular_data_access}, we can retrieve $n^+(u)$ using 4 cache lines, but if $u$'s neighbours were packed closely together, it would be possible to retrieve all of their data using a single cache line.
    \item \textit{Temporal Locality}: a pair of vertices $u, v$ exhibit \textit{temporal} locality if there exist substantial overlap between $n(u)$ and $n(v)$. For example, $u, v$ exhibit perfect temporal locality if $n(u) = n(v)$.
\end{enumerate}
\algnewcommand{\LineComment}[1]{\State \(\triangleright\) #1}
\begin{algorithm}
    \begin{algorithmic}[1]
        \For{$u \in V$:}
        \For{$v \in n^{+}(u)$:}
        \State \texttt{work}$(\texttt{v\_data}[v])$
        \EndFor
        \EndFor
    \end{algorithmic}
    \caption{Out-neighbourhood Graph Traversal}
    \label{alg:algorithm1}
\end{algorithm}

\begin{figure}[!htb]
    \begin{adjustwidth}{-0.75in}{-0.75in}
        \centering
        \includesvg[width=6in]{./ipe_plots/irregular_data_access.svg}
        \caption{
            Irregular access to the \texttt{v\_data} array by traversing the out-neighbourhood of $u$. 
            The circled numbers refer to the order in which elements from \texttt{v\_data} are retrieved. 
            Due to the distance between the IDs of $u$'s out-neighbours, this traversal suffers from poor \textit{spatial} locality. Figure adapted from \cite{lwr}.
        }
        \label{fig:irregular_data_access}   % label should change
    \end{adjustwidth}
\end{figure}

The issue of poor locality is exacerbated in the context of modern Graph Processing Systems (\ac{GPS}) that use multiple threads to iterate over the graph concurrently. Modern architectures use a hierarchical memory system consisting of a smaller private L1 and L2 cache per core, and larger shared LLC cache. Parallel iteration over the vertices of the graph (as in Algorithm \ref{alg:par_graph_traverse}) is usually done by modern \ac{GPS} by picking 
a \textit{Vertex Partitioning} strategy and \textit{Mode of Computation}. 

Two common\textit{Vertex Partitioning}
The two conventional \textit{Modes of Computation} for parallel graph processing are either \textit{Push} or \textit{Pull}.

Regardless of the choice of partitioning and mode of computation, parallel graph algorithms generally suffer from poor locality. As threads iterate over the neighbourhoods of vertices, there is no guarantee of spatial and/or temporal locality being shared between neighbourhoods or threads. Since threads will be operating on non-overlapping regions of \texttt{v\_data}, it is likely they will compete for the capacity of the LLC, in turn causing a larger number of Cache Evictions and Misses. 
% As a result, the shared cache is likely to be competitively  are likely to quickly fill 

\begin{algorithm}
    \begin{algorithmic}[1]
        \ParFor{$u \in V$:}
            \For{$v \in n^+(u)$:}
            \State \texttt{work}$(\texttt{v\_data}[v])$
            \EndFor
        \EndParFor
        
    \end{algorithmic}
    \caption{Parallel neighbourhood Graph Traversal}
    \label{alg:par_graph_traverse}
\end{algorithm}


\subsection{Adjacency Matrix}
\subsection{Adjacency List}
\subsection{Compressed Representations}
\subsubsection{Compressed Sparse Row, Column}
\subsection{Memory Access Patterns and Bottlenecks in Graph processing}

\section{Graph Ordering}
% In an attempt to alleviate the issues described above, a wealth of previous work has gone into research
% about Graph Ordering Techniques. These techniques reorder (i.e. relabel) either the vertices or edges of the graph to improve the memory access locality exhibited when traversing over the graph. First, we define spatial locality, and show how vertex reordering can be leveraged to improve it. We also introduce the concepts of light, mid, and heavyweight vertex reordering techniques, and list related examples. Second,
% we define temporal locality, and show how edge reordering can be leveraged to improve it. Finally, we introduce the \ac{HSFC}, and describe its advantages in the context of improving temporal locality of edge traversals.

To address the problems mentioned above, there has been a significant amount of research on Graph Ordering Techniques. These techniques rearrange either the vertices or edges of a graph to enhance the memory access locality during graph traversal. We first define spatial locality and demonstrate how vertex reordering can improve it. We also categorize vertex reordering techniques into light, mid, and heavyweight techniques and provide examples. Next, we define temporal locality and explain how edge reordering can optimize it. Finally, we introduce the Hilbert Space Filling Curve and highlight its benefits in improving the temporal locality of edge traversals

\subsection{Spatial Graph Transforms: Vertex Reordering}
\subsubsection{LightWeight, HeavyWeight, MidWeight, Slashburn}
\subsection{Temporal Graph Transforms: Edge Reordering}
\subsubsection{Hilbert Curve}

\section{PageRank}
% Next, we discuss the PageRank algorithm and challenges that arise when attempting to parallelize it. 
% We introduce the algorithm and reason about why it became such a ubiqutous graph processing benchmark.
% We differentiate between the two computational modes of the algorithm. Finally, we divert our attention to challenges that arise in the context of parallelization. Finally, we introduce Propagation Blocking, a recent optimization used to improve the spatial locality of parallel PageRank computation.

In the next section, we examine the PageRank algorithm and the difficulties associated with parallelizing it. We present the algorithm and explore its widespread use as a benchmark in graph processing. We distinguish between the two modes of computation that can be used to compute a graph's PageRank algorithm. Then, we focus on the challenges in parallelizing the algorithm. Finally, we introduce Propagation Blocking, a recent optimization aimed at enhancing the spatial locality in parallel PageRank computation.
\subsection{Pull vs. Push}
\subsection{Parallelizing Computation}
\par{
}
\subsubsection{Propagation Blocking}
